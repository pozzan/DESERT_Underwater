\section{Modules for the application layer}\label{sec:application}

\begin{description}
   \item {\bf Name:} {\tt uwcbr}
   \item {\bf Description:} This module implements a Constant Bit Rate (CBR) packet traffic between a sender and a receiver. The data traffic can be generated either by injecting packets in the network with a constant time period or according to a Poisson process with given mean. A single {\tt uwcbr} module represents a data flow between a pair of nodes: if there are two or more nodes transmitting to the same destination, the latter should have an equal number of {\tt uwcbr} modules, one for each flow. 
   \item {\bf Library name:} {\tt libuwcbr.so}
   \item {\bf Tcl name:} {\tt Module/UW/CBR}.
   \item {\bf Tcl parameters:} 
   \begin{description}
    \item {\tt period\_}: period between two consecutive packet transmissions;
    \item {\tt destPort\_}: {\tt Port} value of the destination;
    \item {\tt destAddr\_}: {\tt IP} of the destination;
    \item {\tt packetSize\_}: size of the packet payload;
    \item {\tt PoissonTraffic\_}: used to enable or disable the generation of traffic according to a Poisson distribution;
    \item {\tt debug\_}: flag to disable or enable debug messages [range in \{0,1\}] (optional, default value = 0);
    \item {\tt drop\_out\_of\_order\_}: flag used to change the behaviour of the module. If equals to 1 the module will accept packets only if in order, otherwise the module will accept them also out of order. (v1.1.0)
   \end{description}
   \item {\bf Tcl commands:}
   \begin{description}
    \item {\tt start}: start sending packets;
    \item {\tt stop}: stop sending packets;
    \item {\tt getrtt}: get the round trip time seen for the last received packet;
    \item {\tt getftt}: get the forward trip time;
    \item {\tt getper}: get the packet error rate;
    \item {\tt getthr}: get the throughput (in bytes/s);
    \item {\tt getrttstd}: get the standard deviation of the round trip time;
    \item {\tt getfttstd}: get the standard deviation of the forward trip time;
    \item {\tt getsentpkts}: get the number of {\tt CBR} packets sent;
    \item {\tt getrecvpkts}: get the number of {\tt CBR} packets received;
    \item {\tt getcbrheadersize}: get the size in byte of a {\tt hdr\_cbr} packet;
    \item {\tt sendPkt}: send a single {\tt CBR} packet;
    \item {\tt resetStats}: reset all the statistics;
   \end{description}
   \item {\bf Internal packet headers:} {\tt hdr\_uwcbr}.
   \item {\bf External packet headers:} {\tt hdr\_cmn} (common header from ns2); {\tt hdr\_uwip} (uwip header from DESERT);  {\tt hdr\_uwudp} (uwudp header from DESERT).
   \item {\bf Warnings:} This module requires an {\tt UWUDP} module below. A single {\tt UWCBR} module does not represent an application, but a single flow between two nodes: if there are two nodes transmitting to the same node, the latter should have two {\tt UWCBR modules}, one for each flow.
   \item {\bf Parent Libraries:}
   \begin{description}
   \item {\tt libMiracle.so}
   \item {\tt libuwip}
   \item {\tt libuwudp}
   \end{description}
   \item {\bf Tcl Scripts:} 
   \begin{description}
   \item {\tt samples/basic/uwcbr.tcl}
   \end{description}
\end{description}

\vspace{1 cm}

\begin{description}
    \item {\bf Name:} {\tt uwvbr}
    \item {\bf Description:} This module implements a Variable Bit Rate (VBR) packet traffic between a sender and a receiver. The data packet generation process takes place by switching between two different CBR processes, e.g., having different average packet inter-arrival times. 
    The switch between the processes can be configured by the user by providing the switching epochs. Otherwise, the simulator can be instructed to switch at constant or exponentially distributed intervals. A single {\tt uwvbr} module represents a data flow between a pair of nodes: if there are two or more nodes transmitting to the same node, the latter should have an equal number of {\tt uwvbr} modules, one for each flow.
   \item {\bf Library name:} {\tt libuwvbr.so}
   \item {\bf Tcl parameters:} 
   \begin{description}
    \item {\tt period\_}: period between two consecutive packet transmissions;
    \item {\tt destPort\_}: {\tt Port} value of the destination;
    \item {\tt destAddr\_}: {\tt IP} of the destination;
    \item {\tt packetSize\_}: size of the packet payload;
    \item {\tt PoissonTraffic\_}: flag used to enable or disable the generation of traffic according to a Poisson distribution;
    \item {\tt debug\_}: flag to disable or enable debug messages [range in \{0,1\}] (optional, default value = 0);
    \item {\tt drop\_out\_of\_order\_}: flag used to change the behaviour of the module. If equals to 1 the module will accept packets only if in order, otherwise the module will accept them also out of order. (v1.1.0)
   \end{description}
   \item {\bf Tcl commands:}
   \begin{description}
    \item {\tt start}: start sending packets;
    \item {\tt stop}: stop sending packets;
    \item {\tt getrtt}: get the round trip time seen for last received packet;
    \item {\tt getftt}: get the forward trip time;
    \item {\tt getper}: get the packet error rate;
    \item {\tt getthr}: get the throughput (in bytes/s);
    \item {\tt getrttstd}: get the standard deviation of the round trip time;
    \item {\tt getfttstd}: get the standard deviation of the forward trip time;
    \item {\tt getsentpkts}: get the number of {\tt VBR} packets sent;
    \item {\tt getrecvpkts}: get the number of {\tt VBR} packets received;
    \item {\tt getvbrheadersize}: return the size in byte of a {\tt hdr\_vbr} packet;
    \item {\tt sendPkt}: send a single {\tt VBR} packet;
    \item {\tt resetStats}: reset all the statistics;
   \end{description}
   \item {\bf Internal packet headers:} {\tt hdr\_uwvbr}.
   \item {\bf External packet headers:} {\tt hdr\_cmn} (common header from ns2); {\tt hdr\_uwip} (uwip header from DESERT);  {\tt hdr\_uwudp} (uwudp header from DESERT).
   \item {\bf Warnings:} This module requires an {\tt UWUDP} module below. A single {\tt UWVBR} module does not represent an application, but a single flow between two nodes: if there are two nodes transmitting to the same node, the latter should have two {\tt UWVBR modules}, one for each flow.
   \item {\bf Parent Libraries:}
   \begin{description}
   \item {\tt libMiracle.so}
   \item {\tt libuwip}
   \item {\tt libuwudp}
   \end{description}
   \item {\bf Tcl Scripts:} 
   \begin{description}
   \item {\tt samples/basic/uwvbr.tcl}
   \end{description}
\end{description} 
