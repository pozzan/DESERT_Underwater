\documentclass[12pt]{article}
\usepackage{listings}
\usepackage{url}
\usepackage{listing}

\begin{document}

\begin{Large} \noindent {\bf{Department of Information Engineering (DEI), University of Padua}}\\ \end{Large}
\begin{large} {DESERT Documentation} \end{large}\\
\begin{large} {Doxygen Documentation} \end{large}

\section*{Doxygen Documentation}
\begin{itemize}
	\item This is the official web--page of Doxygen: \url{http://www.stack.nl/~dimitri/doxygen};
	\item At \url{http://www.stack.nl/~dimitri/doxygen/commands.html} you can find the complete list of Doxygen commands and an example for each of them;
	\item For the Doxygen tag to use please look at here \url{http://www.yolinux.com/TUTORIALS/LinuxTutorialC++CodingStyle.html} under the section \emph{dOxygen and C++ Automated Documentation Generation}. Expecially under the section \textbf{dOxygen tag} there a list of that that you should use:
	\begin{itemize}
		\item file
		\item brief
		\item author
		\item version
		\item deprecated
		\item see
		\item param[in]
		\item param[out]
		\item return
		\item bug (I hope that no one will use this tag, but just in case\ldots)
		\item warning
	\end{itemize}
\end{itemize}
\section*{Other}
As example you can take a look at UWSUN protocol.\\
Remember to put the Doxygen commands only in the .h file, in the .cc file put only comments on your code, algorithms, loops, etc. but not Doxygen commands.

	

\end{document}